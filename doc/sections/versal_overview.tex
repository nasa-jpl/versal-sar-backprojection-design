% =============================================================================
\section{AI Engine Overview}
% =============================================================================
\label{sec:overview}


\subsection{Parallelism and Data Transfer}
\label{sec:parallelism_and_data_trans}

% -----------------------------------------------------------------------------
% AI Engine Array
% -----------------------------------------------------------------------------
\begin{figure}[hbtp]
  \begin{center}
    \includegraphics[width=\textwidth]
    {figures/aie_array.png}
  \end{center}
  \caption{AI Engine Array \cite{AMD_Versal_AM009}}
  \label{fig:aie_array}
\end{figure}
% -----------------------------------------------------------------------------

AI Engines are an array of very-long instruction word (VLIW) processors with
single instruction multiple data (SIMD) vector units that are highly optimized
for compute-intensive applications \cite{AMD_Versal_UG1079}. The AI Engine
array supports three levels of parallelism:

\begin{itemize}
    \item \textbf{Instruction Level Parallelism (ILP):} Through the VLIW 
        architecture allowing multiple operations to be executed in a single 
        clock cycle.
    \item \textbf{SIMD:} Through vector registers allowing multiple elements
        to be computed in parallel.
    \item \textbf{Multicore:} Through the AI Engine array, allowing up to 400
        AI Engine tiles to execute in parallel.
\end{itemize}

Additionally, the AI Engine tiles provide multiple mechanisms to store,
pipeline, and move data. Figure~\ref{fig:aie_array} highlights several of these
data-transfer mechanisms. The following list gives a more complete overview:

\begin{itemize}
  \item \textbf{Local memory} - Each tile has 32\,KB of data memory for runtime
      buffers.
  \item \textbf{Neighbor memory sharing} - A tile can directly access adjacent
      tiles’ memories with hardware locks for coordination, effectively
      expanding local storage~\cite{AMD_Versal_UG1079}.
  \item \textbf{Cascade ports} - Dedicated links between adjacent AI Engine
      tiles’ multiply-accumulate (MAC) accumulators used to pass partial sums
      or intermediate results. This path bypasses AXI4-Stream and local memory
      for low-latency chained computations.
  \item \textbf{AXI4-Stream (tile-to-tile)} - General-purpose streaming
      interfaces inside the AI Engine array (typically 32 bits per port) with
      handshake signals (\textit{TVALID}, \textit{TREADY}, and often
      \textit{TLAST} for frame boundaries).
  \item \textbf{PLIO ports} - AXI4-Stream connections between the AI Engine
      array and the Programmable Logic (PL); commonly 32, 64, or 128 bits wide
      at the interface boundary.
  \item \textbf{GMIO ports} - Interfaces that move data between external DDR
      (via the NoC) and the AI Engine array. Support burst transfers (e.g.,
      64/128/256\,B).
  \item \textbf{Memory-mapped access (NoC)} - For control or lower-bandwidth
      data movement, the NoC can read/write local memories and configure DMA
      descriptors using AXI memory-mapped transactions.
\end{itemize}

\subsection{AI Engine Kernels}
\label{subsec:aie_kernels}


\subsubsection{Overview}
\label{subsubsec:overview}

An AI Engine kernel is a C/C++ function written with the AI Engine API that
targets the VLIW vector processor in each AI Engine tile. Arrays range from
tens to hundreds of tiles depending on the device; for example, the AI Core
VC1902 integrates 400 AI Engine tiles~\cite{AMD_Versal_DS946}. Kernels are
compiled with the Vitis AI Engine compiler into ELF images, which the graph
runtime loads onto the assigned tiles. Based on the user-specified runtime
ratio and available resources, the compiler may co-locate two kernels to
time-share a tile. If not, it assigns one kernel to the tile.

\subsubsection{SIMD}
\label{subsubsec:simd}

Although most conventional C/C++ will compile, kernels usually need
restructuring to expose vector and instruction-level parallelism. In practice
we express arithmetic with AI Engine intrinsics such as \verb|aie::add()|,
\verb|aie::sub()|, and \verb|aie::mul()|, while local data storage and
retrieval is handled explicitly with vector stores and loads (e.g.,
\verb|aie::store_v<N>()| and \verb|aie::load_v<N>()|)~\cite{AMD_Versal_UG1079}.
These intrinsics convey SIMD intent, operate on vector types held in the tile’s
vector registers, and help the compiler generate packed stores/loads and fused
multiply–accumulate sequences. Without this structure, code is often scalarized
by the compiler, underutilizing the vector hardware.

\subsubsection{Pipelining and Data Movement}
\label{subsubsec:pipelining_and_data_movement}


Loop-level pipelining in an AI Engine kernel relies on careful loop structure
and buffering. For example, using ping-pong (double) buffers or circular
buffers allows one memory block to be used for computing while the other is
loaded or stored, effectively overlapping memory access with computation. When
buffers are aligned and accessed contiguously, the compiler can issue one
vector multiply–accumulate per cycle (subject to memory-port bandwidth) by
preloading operands for the next iteration while writing results from the prior
one. Developers can also guide the compiler with the
\verb|chess_prepare_for_pipelining| directive, which prompts the AI Engine
compiler to create an overlapped (software-pipelined)
schedule~\cite{AMD_Versal_UG1079}. By contrast, High-Level Synthesis (HLS)
pipeline pragmas like \texttt{\#pragma HLS PIPELINE II=1} target PL kernels and
are not used for AI Engine kernels. These loop and buffering patterns let the
compiler overlap loads, computation, and stores; without such structuring,
kernels often stall on memory rather than keeping the AI Engine’s resources
busy.


\subsubsection{Buffer and Stream Interfaces}
\label{subsubsec:buffer_and_stream_inf}

AI Engine kernels can exchange data through either buffer ports or stream
ports. A buffer port maps to a contiguous block of local tile memory that
serves as the kernel’s input or output. A synchronous buffer port blocks until
the entire block is available (for inputs) or consumed (for outputs) before the
kernel proceeds~\cite{AMD_Versal_UG1079}. If multiple synchronous input buffers
are declared, the kernel will not fire until all are ready. By contrast, an
asynchronous buffer port (declared with \texttt{async} in the buffer type name)
does not automatically block. Instead, the kernel code explicitly locks and
releases the buffer to control when data is accessed, giving the designer more
flexibility but also the responsibility for synchronization.

Alternatively, a kernel can use AXI-stream interfaces for continuous data flow.
With streams, the kernel can keep reading or writing beats of data without
waiting for a full block. AI Engine graphs support bi-directional stream ports
(and dedicated cascade streams between neighbors) that run at full
bandwidth~\cite{AMD_Versal_UG1079}. In effect, streaming ports create an
endless data pipeline: as soon as data is available on the stream, the kernel
can process it beat-by-beat. For example, a PL-side AXI stream input can feed
into the AI Engine via a PLIO port, and the kernel’s code can simply advance a
stream iterator each cycle.

\subsection{AI Engine Graphs}
\label{subsec:aie_graphs}

An AI Engine program is organized around a dataflow graph written in C++ with
the Adaptive Data Flow (ADF) API. This graph defines kernels, ports, and the
connections between them (Figure~\ref{fig:aie_graph}). The ADF graph can
interface with the programmable logic (PL), global memory, and the host
processor using specialized ports, such as:

\begin{itemize}
\item \textit{input\_plio} and \textit{output\_plio}: AXI-stream connections to
    and from the PL.
\item \textit{input\_gmio} and \textit{output\_gmio}: memory-mapped DMA
    connections to and from global memory (e.g., DDR).
\item Runtime Parameter (RTP): passes runtime parameter data to AI Engine
    kernels (primarily for passing single values or small arrays).
\end{itemize}

At runtime, the Vitis toolchain manages data movement across these connections.
For buffer connections, it automatically implements double-buffering (e.g.,
ping–pong buffers) so that kernels can compute on one block while the next
block is being filled. For stream connections, data flows continuously without
explicit buffering.

Graph execution follows a Kahn process-network model where each kernel fires
only when its inputs are ready, making kernel execution data-driven. This
enables concurrency across the graph where one kernel may compute while another
loads the next data block and a third writes results. Together with loop-level
pipelining inside kernels, graph-level concurrency maximizes throughput across
the AI Engine array.

\newpage

\subsection{AI Engine Specification Quick Reference}
\label{subsec:quick_ref}

\begin{itemize}[parsep=1em]
    \item GMIO Ports (DDR $\leftrightarrow$ NoC $\leftrightarrow$ AI Engine)
        \begin{itemize}[topsep=0em]
            \item 32 GMIO input ports at the graph/system level
            \item 32 GMIO output ports at the graph/system level
            \item Common NoC burst sizes: 64B, 128B, or 256B.
        \end{itemize}

    \item PLIO Ports ($PL \leftrightarrow AI Engine$)
        \begin{itemize}[topsep=0em]
            \item Configurable bus widths: 32b, 64b, or 128b.
            \item 32 bits per AI Engine clock (i.e. each 32b take 1 clk cycle
                for AI Engine)
        \end{itemize}

    \item Clocks
        \begin{itemize}[topsep=0em]
            \item AI Engine clock: approximately 1.25 GHz (device and
                speed-grade dependent).
            \item NoC clock: approximately 1.0 GHz.
            \item PL clocks: approximately 500 MHz (but can vary widely).
            \item Arm Cortex-A72: approximately 1.35 GHz (also
                device-dependent).
        \end{itemize}

    \item AI Engine Tile Local Data Memory
        \begin{itemize}[topsep=0em]
            \item 32 KB total per tile for runtime data:
                \begin{itemize}[topsep=0em]
                    \item Ping-pong buffers
                    \item Stack, heap, local variables
                    \item Sync locks, ring buffers, etc.
                \end{itemize}
            \item Realistically, a ping-pong buffer can be up to 16 KB each if
                using double-buffering (i.e. ping-pong buffers).
        \end{itemize}

    \item Neighbor Memory Sharing
        \begin{itemize}[topsep=0em]
            \item Each tile can read/write memory of up to 3 adjacent tiles,
                giving a total potential local + neighbor memory of up to 128
                KB (4× 32 KB) in a 2×2 region.
            \item Requires explicit lock-based
                coordination \cite{AMD_Versal_UG1079}.
        \end{itemize}

    \item Tile Program Memory
        \begin{itemize}[topsep=0em]
            \item 16 KB for application program memory size (separate from the
                32 KB data memory).
        \end{itemize}

    \item AI Engine AXI4 Stream I/O
        \begin{itemize}[topsep=0em]
            \item Up to 2 input and 2 output streaming ports, each 32 bits
                wide, 1 word/cycle.
            \item Internal FIFOs enable packing 128 bits every 4 cycles,
                depending on external interface needs.
        \end{itemize}
\end{itemize}

These characteristics apply to AI Engine (V1) devices, not the newer AI
Engine-ML variants, which differ in memory size, clock domains, and additional
ML extensions.


