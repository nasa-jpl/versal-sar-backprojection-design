% =============================================================================
\section{Future Work}
% =============================================================================
\label{sec:future_work}

% =============================================================================
\subsection{Spotlight Mode}
% =============================================================================
\label{sec:spotlight_mode}

The current implementation of spotlight-mode backprojection on the Versal
platform processes only a limited segment of the synthetic aperture,
corresponding to approximately 5 degrees of azimuthal coverage around the scene
center (roughly 602 pulses, where one degree corresponds to about 117 pulses).
This setup provides a focused demonstration of the imaging pipeline but does
not yet encompass the full 360-degree aperture required for complete scene
reconstruction.

Extending this implementation to full circular coverage will primarily involve
modifications on the ARM processor side, which manages data transfers to the AI
Engine kernels. The overall hardware architecture is expected to support
continuous aperture processing; however, the mechanisms for sequentially
feeding multiple data segments and handling intermediate synchronization have
not yet been developed or validated. This functionality was deferred due to
project time and budget constraints but remains a key next step toward
achieving end-to-end spotlight-mode imaging on the Versal platform.

% =============================================================================
\subsection{Stripmap Mode}
% =============================================================================
\label{sec:stripmap_mode}

In spotlight mode, the antenna beam is continuously steered to focus on a fixed
area of interest, producing high-resolution imagery over a relatively small
footprint. In contrast, stripmap mode maintains a fixed antenna pointing
direction and constant azimuth beamwidth as the platform moves along its flight
path. This configuration results in a wider swath of coverage, generating
continuous imagery along the nadir track. Implementing stripmap-mode
backprojection on the Versal platform will require adapting the current
architecture to handle continuous target pixel input across a dynamic scene
geometry as the satellite (or aircraft) advances. This should be feasible with
the current architecture; however, a new dataset that provides the appropriate
input geometry and acquisition parameters for stripmap mode will be required.

% =============================================================================
\subsection{ILP Optimizations}
% =============================================================================
\label{sec:ilp_optimizations}

Section~\ref{sec:overview} outlines three primary optimization mechanisms
available on the Versal platform: ILP, SIMD, and multicore processing. The
existing design relies heavily on SIMD and multicore parallelism, with ILP
contributing to a smaller degree. In the current implementation, ILP
optimizations are applied to select \texttt{for} loops within the AI Engine
code using the \texttt{chess\_prepare\_for\_pipelining} directive, which
enables the compiler to perform pipelining of loop iterations. Further analysis
through profiling tools such as \textit{Vitis Analyzer} is recommended to
better understand how these ILP optimizations affect overall performance and to
identify additional opportunities for improvement. On the PL side, similar
benefits can be achieved by employing HLS directives such as \texttt{\#pragma
HLS PIPELINE} (to reduce the initiation interval), \texttt{UNROLL}, and
\texttt{DATAFLOW}, which together enable concurrent loop iteration and task
execution within synthesized hardware.

% =============================================================================
\subsection{DMA Stride Controller PL Kernel Improvements}
% =============================================================================
\label{sec:dma_strd_ctlr_pl_kern_improve}

The current DMA Stride Controller PL kernel is limited to supporting a single
output stream, with \texttt{AIE\_SWITCHES} fixed at a value of one. The
kernel's top-level function signature, shown below, defines
\texttt{pl\_stream\_out} as an array of AXI4-Stream interfaces:

\begin{lstlisting}[
  language=C++,
  numbers=left,          % or right
  numberstyle=\tiny,     % font size for numbers
  numbersep=8pt,         % gap between numbers and code
  stepnumber=1,          % number every line (2 = every other line)
  firstnumber=1,         % starting number
  xleftmargin=2em,       % add left margin so numbers don’t collide
  framexleftmargin=2em,  % keep frame aligned if you use frame=single
  basicstyle=\ttfamily\footnotesize,
  caption={Function signature of the DMA Stride Controller PL kernel},
  captionpos=b
]
int dma_stride_controller(
    ap_uint<64>* ddr_mem, 
    hls::stream<ap_axiu<128, 0, 0, 0>> pl_stream_out[AIE_SWITCHES]
) {
    #pragma HLS INTERFACE axis port=pl_stream_out
    #pragma HLS INTERFACE m_axi port=ddr_mem offset=slave bundle=gmem \
depth=PULSES*AZ_SAMPLES*RC_SAMPLES*2
    #pragma HLS INTERFACE s_axilite port=ddr_mem bundle=control
    #pragma HLS INTERFACE s_axilite port=return  bundle=control
}
\end{lstlisting}

While simulation and testbench results confirm that configurations with
\texttt{AIE\_SWITCHES > 1} can functionally operate, integration within the
full Versal system is currently constrained by the Vitis toolchain.
Specifically, the \texttt{system.cfg} file---used to describe PL-to-AIE
connectivity---cannot adequately reference arrayed AXI4-Stream ports such as
\texttt{pl\_stream\_out[AIE\_SWITCHES]}. At present, the toolchain requires
explicit, uniquely named stream interfaces for each connection, preventing
parameterized array references at synthesis or link time.

This limitation reduces the degree of AI Engine parallelism that can be
utilized, since only one \texttt{bpCluster} can be driven at a time. Future
work should address this constraint by exploring one of the following paths: 

\begin{itemize}
    \item \textbf{Kernel Prototyping:} Create multiple kernel prototypes that
        explicitly instantiate separate \texttt{pl\_stream\_out\_N} ports for
        each target number of \texttt{bpCluster}'s (or \texttt{AIE\_SWITCHES}).
    \item \textbf{Kernel Replication:} Redesign the DMA Stride Controller to
        operate with a single cluster output stream and instantiate multiple
        kernel instances, one per AI Engine cluster.
    \item \textbf{Toolchain Integration:} Investigate updates or workarounds in
        newer Vitis releases that support arrayed AXI4-Stream interfaces in
        \texttt{system.cfg} connectivity specifications.
\end{itemize}

Expanding the DMA Stride Controller to handle multiple concurrent streams would
enable the full utilization of the available AI Engine compute resources,
allowing simultaneous data transfers to multiple \texttt{bpCluster} and
significantly improving throughput. Work on this front was deferred due to
funding and time constraints but remains a key area for enhancing system
scalability and performance when using the \textit{pl\_stride} branch.

% =============================================================================
\subsection{DSP Utilization}
% =============================================================================
\label{sec:dsp_utilization}

Future work will focus on extending the current implementation to leverage the
Versal DSP engines for radar signal preprocessing tasks such as FFT and iFFT
for pulse compression. Although these operations were initially implemented and
verified within the AI Engine during early development, they were later removed
to prioritize integrating the core components of the backprojection algorithm
for image focusing. Since FFT and iFFT transforms are highly regular,
arithmetic-intensive operations, they can be executed efficiently within either
the PL or dedicated DSP blocks, freeing the AI Engine for more complex
data-dependent computations.

In this work, the MATLAB reference implementation generated post-processed,
iFFT time-domain phase history data to serve as input to the Versal hardware.
This allowed resolution adjustments through zero-padding in the frequency
domain to produce input data that was compatible with the AI Engine architected
design and eliminated the need for on-chip frequency-domain processing during
evaluation.

In a complete radar signal chain, pulse compression requiring both FFT and iFFT
must occur on-board as part of the real-time data path. Offloading these
transforms to the Versal DSP engines would more accurately represent
operational radar processing and improve system-level efficiency by balancing
the computational load across all available resources.

% =============================================================================
\subsection{AI Engine Algorithm Optimizations}
% =============================================================================
\label{sec:aie_algo_optimization}

Several opportunities exist to further optimize the AI Engine implementation
for higher performance and efficiency.

\newcommand{\heading}[1]{%
  \par\vspace{0.8\baselineskip}% space above
  \noindent\textbf{#1}\par\vspace{0.45\baselineskip}% space below
}

\heading{Internal Target Pixel Generation via RTP Configuration}

Currently, target pixels are sent from DDR to the AI Engine Image
Reconstruction kernels. It would be more efficient to configure the kernels
using RTPs to internally generate target pixels. This approach would eliminate
the need to stream target pixel data from DDR, reducing both latency and
bandwidth utilization.  Early in development, this method was explored but was
later replaced with the processor-driven approach for greater flexibility. Now
that the full system is operational, revisiting this method may yield
significant performance improvements with potentially minor architectural
changes.\par\medskip

\heading{Selective RC Sample Distribution}

At present, all range-compressed samples are broadcast to every image
reconstruction kernel, even when only a subset is needed. A more efficient
approach would send only the required RC segments to each kernel, freeing local
buffer space and enabling denser placement of kernels within the AI Engine
array.  The RC sample indices used for interpolation (as described in
Section~\ref{subsec:img_recon_kern}, Step~\ref{itm:interp}) depend on the
target pixels and slow-time data. To implement this optimization, the ARM
processor could pre-compute which RC segments are required by each kernel and
stream them selectively, potentially through the Pixel Demux kernel rather than
the Data Broadcast kernel. Alternatively, a dedicated AI Engine kernel could
handle RC segment distribution.  However, if the first item above is
implemented (where target pixels are generated internally), the processor would
no longer know which RC segments each kernel requires, increasing coordination
complexity. The optimal implementation would depend on data accessibility and
synchronization between the processor and AI Engine.

\heading{Dynamic Buffer Indexing for Interpolation}

As described in Section~\ref{subsec:img_recon_kern} (Step~\ref{itm:interp}),
the interpolation step currently performs element-by-element access into the RC
buffer because the required indices are dynamically generated at runtime. Since
these indices are non-contiguous and cannot be known ahead of time, SIMD
vectorization cannot be applied, and the kernel must fall back to a scalar loop
for indexing. This breaks the otherwise continuous flow of vectorized
computation and introduces a notable performance penalty, as each image
reconstruction kernel must perform these lookups for every pulse. Future work
should revisit this section of the algorithm to determine whether the indexing
method can be restructured or optimized, potentially leveraging updates to the
AI Engine API or library support in future AMD tool releases.

% =============================================================================
\subsection{Comparison with XD100 AMD Reference Design}
% =============================================================================
\label{sec:xd100_comparison}

In August 2025, AMD released the XD100 reference design and tutorial series for
AI Engine development, which includes a dedicated section on backprojection for
SAR~\cite{AMD_Versal_XD100}. Their implementation, titled
\textit{Back-Projection for Synthetic Aperture Radar on AI Engines}, uses the
same GOTCHA Volumetric SAR dataset employed in this work and demonstrates a
fully realized backprojection pipeline executing on the AI Engine array. The
tutorial showcases system-level concepts such as multi-rate scheduling, dynamic
range management, and vectorized implementations of \texttt{sin()},
\texttt{cos()}, and \texttt{sqrt()} using the Vitis DSP Library.

This reference design was released near the conclusion of the present effort,
after development resources had been exhausted. Although both efforts address
the same algorithm and dataset, the two designs take different architectural
approaches that would be valuable to analyze in greater detail. A direct
comparison of throughput, tile utilization, and overall energy efficiency could
provide meaningful insight into the relative performance and scalability of
each implementation. However, such an analysis was outside the current project
scope due to time and funding constraints and remains an important direction
for future investigation.
