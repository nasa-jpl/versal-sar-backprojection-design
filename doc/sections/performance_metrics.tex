% =============================================================================
\section{Performance Metrics}
% =============================================================================
\label{sec:perf_metrics}

The following sections present performance, power, and architecture metrics for
different build configurations and AIE graph architectures. All configurations
are derived from the same stepped-frequency X-band radar dataset. The waveform
and frequency parameters are therefore fixed across all tests and are
summarized in Table~\ref{tab:global_radar_params}. These constants define the
underlying radar physics, while configuration-specific parameters (e.g., number
of pulses, range-compressed samples, and aperture size) vary per experiment and
are reported in the subsequent subsections.

\begin{table}[hbt]
  \centering
  \caption{Global radar characteristics (constant for all configurations).}
  \label{tab:global_radar_params}
  \begin{tabular}{@{}ll@{}}
    \toprule
    \textbf{Parameter} & \textbf{Value} \\
    \midrule
    Physical frequency samples ($N_f$) & 424 \\
    Frequency step ($\Delta f$) & 1.4713026 MHz \\
    Start frequency ($f_{\min}$) & 9.2880804 GHz \\
    Stop frequency ($f_{\max}$)  & 9.9119127 GHz \\
    Center frequency ($f_0$)     & 9.5999966 GHz \\
    Wavelength ($\lambda = c/f_0$) & 3.12 cm \\
    Polarization & HH \\
    Dataset source & GOTCHA (pass 1) \\
    \bottomrule
  \end{tabular}
\end{table}

All timing measurements are captured using \texttt{CLOCK\_MONOTONIC} on the ARM
Cortex-A72 processor. Estimated power consumption was derived using the
\gls{xpe} tool and actual power draw was captured from INA226 current-sense
monitors on the VCK190 board.

\newpage

% =============================================================================
\subsection{Test \#1: 425x602, 5.1° Az Coverage, \textit{main} branch}
% =============================================================================
\label{subsec:config1}

\subsubsection*{Configuration}

\begin{table}[hbt]
  \centering
  \caption{Configuration parameters for Test \#1.}
  \label{tab:config1_params}
  \begin{tabular}{@{}ll@{}}
    \toprule
    \textbf{Parameter} & \textbf{Value} \\
    \midrule
    Pulses processed (\texttt{PULSES}) & 602 \\
    Range-compressed samples (\texttt{RC\_SAMPLES}) & 512 \\
    Total aperture angle ($\Theta$) & 0.0895 rad ($\approx$ 5.1$^{\circ}$) \\
    \midrule
    \textbf{Scene coverage} & 101.88 m (range) x 108.41 m (cross-range) \\
    \textbf{Image resolution} & 0.24 m (range) x 0.18 m (cross-range) \\
    \textbf{Resolution cells (approx.)} & 425 (range) x 602 (cross-range) \\
    \bottomrule
  \end{tabular}
\end{table}

\FloatBarrier
\subsubsection*{Focused Image}

\begin{figure}[!htb]
  \centering
  \begin{subfigure}{0.60\linewidth}
    \centering
    \includegraphics[width=\linewidth, trim={0 20bp 0 10bp},
    clip]{figures/test1/versal_focused_img.jpg}
    \caption{Versal generated image of 425x602 resolution over 5.1° synthetic
    aperture for Test \#1.}
    \label{fig:versal_img_test1}
  \end{subfigure}\vfill
  \begin{subfigure}{0.60\linewidth}
    \centering
    \includegraphics[width=\linewidth, trim={0 20bp 0 10bp},
    clip]{figures/MATLAB_GOTCHA_5Deg_512Nfft.jpg}
    \caption{MATLAB generated image of 425x571 resolution over 5.0° synthetic
    aperture (\texttt{FileNum = 5}, \texttt{data.Nfft = 512}).}
    \label{fig:matlab_img_test1}
  \end{subfigure}
  \label{fig:matlab_versal_side_by_side}
\end{figure}

\newpage

\FloatBarrier
\subsubsection*{Performance}

\begin{table}[hbt]
  \centering
  \caption{Segment execution times measured on ARM Cortex-A72 processor.}
  \label{tab:perf_times}
  \begin{tabular}{
      l  % Segment
      l  % Domain
      r  % Time (ms)
  }
    \toprule
    {Segment} & {Domain} & {Time (ms)} \\
    \midrule
    Load xclbin, instantiate AIE graph handles, and malloc memory & ARM & 1024.31         \\
    Populating data buffers                                       & ARM & 2054203.71      \\
    Generating target pixels                                      & ARM & 10.70           \\
    Run AIE graphs                                                & ARM & 0.40            \\
    Perform Backprojection                                        & AIE & \textsuperscript{*}850.53 \\
    \midrule
    \textbf{Total (HOST)} &  & \textbf{2055237.12} \\
    \textbf{Total (AIE)}  &  & \textsuperscript{*}\textbf{850.53}     \\
    \bottomrule
  \end{tabular}
  \vspace{0.5em}
  \begin{minipage}{0.9\linewidth}
    \footnotesize
    *Execution times were measured while the INA226 power-logging process was
    running at 10~samples/s to capture transient AIE utilization. Because
    timing was recorded on the ARM processor, these measurements can include
    small variations from shared CPU scheduling between the timer start/stop
    calls and the INA226 logging task. In some runs, the \textit{Perform
    Backprojection} segment completed in as little as 759.05~ms.
  \end{minipage}
\end{table}

\noindent\textbf{NOTE:} \textit{Populating data buffers} time segment takes
approximately 35 minutes due to loading the slow time and range-compressed
samples into DDR from a large CSV file. During normal operation, this data
would be streamed in through a high-speed interface.

\FloatBarrier
\subsubsection*{Power}

\begin{figure}[hbt]
  \centering
  \includegraphics[width=0.9\linewidth]{figures/test1/ina226_power.png}
  \caption{Measured power (W) during Test \#1 execution.}
  \label{fig:ina226_power_test1}
\end{figure}

\begin{figure}[hbt]
  \centering
  \includegraphics[width=0.9\linewidth]{figures/test1/ina226_current.png}
  \caption{Measured current (mA) during Test \#1 execution.}
  \label{fig:ina226_current_test1}
\end{figure}

\begin{figure}[hbt]
  \centering
  \includegraphics[width=0.9\linewidth]{figures/test1/ina226_voltage.png}
  \caption{Measured voltage (mV) during Test \#1 execution.}
  \label{fig:ina226_voltage_test1}
\end{figure}

\begin{figure}[hbt]
  \centering
  \includegraphics[width=\linewidth]{figures/test1/xpe.png}
  \caption{Theoretical power breakdown from XPE for Test \#1.}
  \label{fig:xpe_test1}
\end{figure}

\begin{figure}[hbt]
  \centering
  \includegraphics[width=\linewidth]{figures/test1/xpe_graphs.png}
  \caption{Theoretical power and static current from XPE for Test \#1.}
  \label{fig:xpe_graph_test1}
\end{figure}

\FloatBarrier
\subsubsection*{Kernel and Dataflow Architecture}

\begin{figure}[hbt]
  \centering
  \includegraphics[width=\linewidth]{figures/test1/kernel_tile_layout.png}
  \caption{Kernel placement in the AI Engine array.
  Each \texttt{bp\_cluster} is confined to a 8x7 grid.}
  \label{fig:kernel_tile_layout_test1}
\end{figure}

\begin{figure}[hbt]
  \centering
  \includegraphics[width=\linewidth]{figures/test1/kernel_tile_dataflow.png}
  \caption{Kernel-to-kernel dataflow connections in AI Engine fabric.}
  \label{fig:kernel_tile_dataflow_test1}
\end{figure}

