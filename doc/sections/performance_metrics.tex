% =============================================================================
\section{Performance Metrics}
% =============================================================================
\label{sec:perf_metrics}

The following sections present performance, power, and architecture metrics for
different build configurations and AIE graph architectures. All configurations
are derived from the same stepped-frequency X-band radar dataset. The waveform
and frequency parameters are therefore fixed across all tests and are
summarized in Table~\ref{tab:global_radar_params}. These constants define the
underlying radar physics, while configuration-specific parameters (e.g., number
of pulses, range-compressed samples, and aperture size) vary per experiment and
are reported in the subsequent subsections.

\begin{table}[hbt]
  \centering
  \caption{Global radar characteristics (constant for all configurations).}
  \label{tab:global_radar_params}
  \begin{tabular}{@{}ll@{}}
    \toprule
    \textbf{Parameter} & \textbf{Value} \\
    \midrule
    Physical frequency samples ($N_f$) & 424 \\
    Frequency step ($\Delta f$) & 1.4713026 MHz \\
    Start frequency ($f_{\min}$) & 9.2880804 GHz \\
    Stop frequency ($f_{\max}$)  & 9.9119127 GHz \\
    Center frequency ($f_0$)     & 9.5999966 GHz \\
    Wavelength ($\lambda = c/f_0$) & 3.12 cm \\
    Polarization & HH \\
    Dataset source & GOTCHA (pass 1) \\
    \bottomrule
  \end{tabular}
\end{table}

All timing measurements are captured using \texttt{CLOCK\_MONOTONIC} on the ARM
Cortex-A72 processor. Estimated power consumption was derived using the
\gls{xpe} tool and actual power draw was captured from INA226 current-sense
monitors on the VCK190 board.

\newpage

% =============================================================================
\subsection{Test \#1: 425x602, 5.1° Az Coverage, \textit{main} branch}
% =============================================================================
\label{subsec:test1}

\subsubsection*{Configuration}

\begin{table}[hbt]
  \centering
  \caption{Configuration parameters for Test \#1.}
  \label{tab:test1_params}
  \begin{tabular}{@{}ll@{}}
    \toprule
    \textbf{Parameter} & \textbf{Value} \\
    \midrule
    Pulses processed (\texttt{PULSES}) & 602 \\
    Range-compressed samples (\texttt{RC\_SAMPLES}) & 512 \\
    Total aperture angle ($\Theta$) & 0.0895 rad ($\approx$ 5.1$^{\circ}$) \\
    \midrule
    \textbf{Scene coverage} & 101.88 m (range) x 108.41 m (cross-range) \\
    \textbf{Image resolution} & 0.24 m (range) x 0.18 m (cross-range) \\
    \textbf{Resolution cells (approx.)} & 425 (range) x 602 (cross-range) \\
    \bottomrule
  \end{tabular}
\end{table}

\FloatBarrier
\subsubsection*{Focused Image}

\begin{figure}[!htb]
  \centering
  \begin{subfigure}{0.60\linewidth}
    \centering
    \includegraphics[width=\linewidth, trim={0 20bp 0 10bp},
    clip]{figures/test1/versal_focused_img.jpg}
    \caption{Versal generated image of 425x602 resolution over 5.1° synthetic
    aperture for Test \#1.}
    \label{fig:versal_img_test1}
  \end{subfigure}\vfill
  \begin{subfigure}{0.60\linewidth}
    \centering
    \includegraphics[width=\linewidth, trim={0 20bp 0 10bp},
    clip]{figures/MATLAB_GOTCHA_5Deg_512Nfft.jpg}
    \caption{MATLAB generated image of 425x571 resolution over 5.0° synthetic
    aperture (\texttt{FileNum = 5}, \texttt{data.Nfft = 512}).}
    \label{fig:matlab_img_test1}
  \end{subfigure}
  \label{fig:matlab_versal_side_by_side_test1}
\end{figure}

\newpage

\FloatBarrier
\subsubsection*{Performance}

\begin{table}[hbt]
  \centering
  \caption{Segment execution times measured on ARM Cortex-A72 processor.}
  \label{tab:perf_times_test1}
  \begin{tabular}{
      l  % Segment
      l  % Domain
      r  % Time (ms)
  }
    \toprule
    {Segment} & {Domain} & {Time (ms)} \\
    \midrule
    Load xclbin, instantiate AIE graph handles, and malloc memory & ARM & 1024.31         \\
    Populating data buffers                                       & ARM & 2054203.71      \\
    Generating target pixels                                      & ARM & 10.70           \\
    Run AIE graphs                                                & ARM & 0.40            \\
    Perform Backprojection                                        & AIE & \textsuperscript{*}850.53 \\
    \midrule
    \textbf{Total (ARM)} &  & \textbf{2055237.12} \\
    \textbf{Total (AIE)}  &  & \textsuperscript{*}\textbf{850.53}     \\
    \bottomrule
  \end{tabular}
  \vspace{0.5em}
  \begin{minipage}{0.9\linewidth}
    \footnotesize
    *Execution times were measured while the INA226 power-logging process was
    running at 10~samples/s to capture transient AIE utilization. Because
    timing was recorded on the ARM processor, these measurements can include
    small variations from shared CPU scheduling between the timer start/stop
    calls and the INA226 logging task. In some runs, the \textit{Perform
    Backprojection} segment completed in as little as 759.05~ms.
  \end{minipage}
\end{table}

\textit{Populating data buffers} time segment takes approximately 35 minutes
due to loading the slow time and range-compressed samples into DDR from a large
CSV file. During normal operation, this data would be streamed in through a
high-speed interface.

\FloatBarrier
\subsubsection*{Power}

\begin{figure}[hbt]
  \centering
  \includegraphics[width=0.9\linewidth]{figures/test1/ina226_power.png}
  \caption{Measured power (W) during Test \#1 execution.}
  \label{fig:ina226_power_test1}
\end{figure}

\begin{figure}[hbt]
  \centering
  \includegraphics[width=0.9\linewidth]{figures/test1/ina226_current.png}
  \caption{Measured current (mA) during Test \#1 execution.}
  \label{fig:ina226_current_test1}
\end{figure}

\begin{figure}[hbt]
  \centering
  \includegraphics[width=0.9\linewidth]{figures/test1/ina226_voltage.png}
  \caption{Measured voltage (mV) during Test \#1 execution.}
  \label{fig:ina226_voltage_test1}
\end{figure}

\begin{figure}[hbt]
  \centering
  \includegraphics[width=\linewidth]{figures/test1/xpe.png}
  \caption{Theoretical power breakdown from XPE for Test \#1.}
  \label{fig:xpe_test1}
\end{figure}

\begin{figure}[hbt]
  \centering
  \includegraphics[width=\linewidth]{figures/test1/xpe_graphs.png}
  \caption{Theoretical power and static current from XPE for Test \#1.}
  \label{fig:xpe_graph_test1}
\end{figure}

\FloatBarrier
\subsubsection*{Kernel and Dataflow Architecture}

\begin{figure}[hbt]
  \centering
  \includegraphics[width=\linewidth]{figures/test1/kernel_tile_layout.png}
  \caption{Kernel placement in the AI Engine array.
  Each \texttt{bp\_cluster} is confined to a 8x7 grid.}
  \label{fig:kernel_tile_layout_test1}
\end{figure}

\begin{figure}[hbt]
  \centering
  \includegraphics[width=\linewidth]{figures/test1/kernel_tile_dataflow.png}
  \caption{Kernel-to-kernel dataflow connections in AI Engine fabric.}
  \label{fig:kernel_tile_dataflow_test1}
\end{figure}

\newpage
% =============================================================================
\subsection{Test \#2: 425x723, 6.1° Az Coverage, \textit{host\_stride} branch}
% =============================================================================
\label{subsec:test2}

\subsubsection*{Configuration}

\begin{table}[hbt]
  \centering
  \caption{Configuration parameters for Test \#2.}
  \label{tab:test2_params}
  \begin{tabular}{@{}ll@{}}
    \toprule
    \textbf{Parameter} & \textbf{Value} \\
    \midrule
    Pulses processed (\texttt{PULSES}) & 714 \\
    Range-compressed samples (\texttt{RC\_SAMPLES}) & 512 \\
    Total aperture angle ($\Theta$) & 0.1061 rad ($\approx$ 6.1$^{\circ}$) \\
    \midrule
    \textbf{Scene coverage} & 101.88 m (range) x 108.41 m (cross-range) \\
    \textbf{Image resolution} & 0.24 m (range) x 0.15 m (cross-range) \\
    \textbf{Resolution cells (approx.)} & 425 (range) x 723 (cross-range) \\
    \bottomrule
  \end{tabular}
\end{table}

\FloatBarrier
\subsubsection*{Focused Image}

\begin{figure}[!htb]
  \centering
  \begin{subfigure}{0.58\linewidth}
    \centering
    \includegraphics[width=\linewidth, trim={0 20bp 0 10bp},
    clip]{figures/test2/versal_focused_img.jpg}
    \caption{Versal generated image of 425x723 resolution over 6.1° synthetic
    aperture for Test \#2.}
    \label{fig:versal_img_test2}
  \end{subfigure}\vfill
  \begin{subfigure}{0.58\linewidth}
    \centering
    \includegraphics[width=\linewidth, trim={0 20bp 0 10bp},
    clip]{figures/test2/MATLAB_GOTCHA_6Deg_512Nfft.jpg}
    \caption{MATLAB generated image of 425x723 resolution over 6.0° synthetic
    aperture (\texttt{FileNum = 6}, \texttt{data.Nfft = 512}).}
    \label{fig:matlab_img_test2}
  \end{subfigure}
  \label{fig:matlab_versal_side_by_side_test2}
\end{figure}

\newpage

\FloatBarrier
\subsubsection*{Performance}

\begin{table}[hbt]
  \centering
  \caption{Segment execution times measured on ARM Cortex-A72 processor.}
  \label{tab:perf_times_test2}
  \begin{tabular}{
      l  % Segment
      l  % Domain
      r  % Time (ms)
  }
    \toprule
    {Segment} & {Domain} & {Time (ms)} \\
    \midrule
    Load xclbin, instantiate AIE graph handles, and malloc memory & ARM & 1069.82         \\
    Populating data buffers                                       & ARM & 2261441.89      \\
    Generating target pixels                                      & ARM & 30.24           \\
    Run AIE graphs                                                & ARM & 0.40            \\
    Perform Backprojection                                        & AIE & \textsuperscript{*}979.90 \\
    \midrule
    \textbf{Total (ARM)} &  & \textbf{2262542.35} \\
    \textbf{Total (AIE)}  &  & \textsuperscript{*}\textbf{979.90}     \\
    \bottomrule
  \end{tabular}
  \vspace{0.5em}
  \begin{minipage}{0.9\linewidth}
    \footnotesize
    *Execution times were measured while the INA226 power-logging process was
    running at 10~samples/s to capture transient AIE utilization. Because
    timing was recorded on the ARM processor, these measurements can include
    small variations from shared CPU scheduling between the timer start/stop
    calls and the INA226 logging task. In some runs, the \textit{Perform
    Backprojection} segment completed in as little as 901.38~ms.
  \end{minipage}
\end{table}

Because the ARM processor takes care of the pre-sorting for the AI Engine in
the \textit{host\_stride} branch, the \textit{Generating target pixels} takes
rougly 20ms longer to complete. Additionally, \textit{Populating data buffers}
time segment takes approximately 37 minutes due to loading the slow time and
range-compressed samples into DDR from a large CSV file. During normal
operation, this data would be streamed in through a high-speed interface.


\FloatBarrier
\subsubsection*{Power}

\begin{figure}[hbt]
  \centering
  \includegraphics[width=0.89\linewidth]{figures/test2/ina226_power.png}
  \caption{Measured power (W) during Test \#2 execution.}
  \label{fig:ina226_power_test2}
\end{figure}

\begin{figure}[hbt]
  \centering
  \includegraphics[width=0.89\linewidth]{figures/test2/ina226_current.png}
  \caption{Measured current (mA) during Test \#2 execution.}
  \label{fig:ina226_current_test2}
\end{figure}

\begin{figure}[hbt]
  \centering
  \includegraphics[width=0.89\linewidth]{figures/test2/ina226_voltage.png}
  \caption{Measured voltage (mV) during Test \#2 execution.}
  \label{fig:ina226_voltage_test2}
\end{figure}

\begin{figure}[hbt]
  \centering
  \includegraphics[width=\linewidth]{figures/test2/xpe.png}
  \caption{Theoretical power breakdown from XPE for Test \#2.}
  \label{fig:xpe_test2}
\end{figure}

\begin{figure}[hbt]
  \centering
  \includegraphics[width=\linewidth]{figures/test2/xpe_graphs.png}
  \caption{Theoretical power and static current from XPE for Test \#2.}
  \label{fig:xpe_graph_test2}
\end{figure}

\FloatBarrier
\subsubsection*{Kernel and Dataflow Architecture}

\begin{figure}[hbt]
  \centering
  \includegraphics[width=\linewidth]{figures/test2/kernel_tile_layout.png}
  \caption{Kernel placement in the AI Engine array.
  Each \texttt{bp\_cluster} is confined to a 8x7 grid.}
  \label{fig:kernel_tile_layout_test2}
\end{figure}

\begin{figure}[hbt]
  \centering
  \includegraphics[width=\linewidth]{figures/test2/kernel_tile_dataflow.png}
  \caption{Kernel-to-kernel dataflow connections in AI Engine fabric.}
  \label{fig:kernel_tile_dataflow_test2}
\end{figure}

\newpage
% =============================================================================
\subsection{Test \#3: 425x99, 0.8° Az Coverage, \textit{pl\_stride} branch}
% =============================================================================
\label{subsec:test3}

Recall that the \textit{pl\_stride} branch currnelyt only works with an
\texttt{AIE\_SWITCHES} value of 1, therefore less data can be processesd at any
given time. Future work aims to increase this number (see
Section~\ref{sec:dma_strd_ctlr_pl_kern_improve} for more information).

\subsubsection*{Configuration}

\begin{table}[hbt]
  \centering
  \caption{Configuration parameters for Test \#3.}
  \label{tab:test3_params}
  \begin{tabular}{@{}ll@{}}
    \toprule
    \textbf{Parameter} & \textbf{Value} \\
    \midrule
    Pulses processed (\texttt{PULSES}) & 100 \\
    Range-compressed samples (\texttt{RC\_SAMPLES}) & 512 \\
    Total aperture angle ($\Theta$) & 0.0147 rad ($\approx$ 0.8$^{\circ}$) \\
    \midrule
    \textbf{Scene coverage} & 101.88 m (range) x 108.41 m (cross-range) \\
    \textbf{Image resolution} & 0.24 m (range) x 1.10 m (cross-range) \\
    \textbf{Resolution cells (approx.)} & 425 (range) x 99 (cross-range) \\
    \bottomrule
  \end{tabular}
\end{table}

\FloatBarrier
\subsubsection*{Focused Image}

\begin{figure}[!htb]
  \centering
  \begin{subfigure}{0.52\linewidth}
    \centering
    \includegraphics[width=\linewidth, trim={0 20bp 0 10bp},
    clip]{figures/test3/versal_focused_img.jpg}
    \caption{Versal generated image of 425x99 resolution over 0.8° synthetic
    aperture for Test \#3.}
    \label{fig:versal_img_test3}
  \end{subfigure}\vfill
  \begin{subfigure}{0.52\linewidth}
    \centering
    \includegraphics[width=\linewidth, trim={0 20bp 0 10bp},
    clip]{figures/test3/MATLAB_GOTCHA_1Deg_512Nfft.jpg}
    \caption{MATLAB generated image of 425x117 resolution over 1.0° synthetic
    aperture (\texttt{FileNum = 1}, \texttt{data.Nfft = 512}).}
    \label{fig:matlab_img_test3}
  \end{subfigure}
  \label{fig:matlab_versal_side_by_side_test3}
\end{figure}

\newpage

\FloatBarrier
\subsubsection*{Performance}

\begin{table}[hbt]
  \centering
  \caption{Segment execution times measured on ARM Cortex-A72 processor.}
  \label{tab:perf_times_test3}
  \begin{tabular}{
      l  % Segment
      l  % Domain
      r  % Time (ms)
  }
    \toprule
    {Segment} & {Domain} & {Time (ms)} \\
    \midrule
    Load xclbin, instantiate AIE graph handles, and malloc memory & ARM & 539.87    \\
    Populating data buffers                                       & ARM & 353081.47 \\
    Generating target pixels                                      & ARM & 301.64    \\
    Run AIE graphs                                                & ARM & 0.17      \\
    Perform Backprojection                                        & AIE & 129.95    \\
    \midrule
    \textbf{Total (ARM)} &  & \textbf{353923.15} \\
    \textbf{Total (AIE)}  &  & \textbf{129.95} \\
    \bottomrule
  \end{tabular}
\end{table}

\textit{Populating data buffers} time segment takes approximately 6 minutes due
to loading the slow time and range-compressed samples into DDR from a large CSV
file. During normal operation, this data would be streamed in through a
high-speed interface.

\FloatBarrier
\subsubsection*{Power}

\begin{figure}[hbt]
  \centering
  \includegraphics[width=0.9\linewidth]{figures/test3/ina226_power.png}
  \caption{Measured power (W) during Test \#3 execution.}
  \label{fig:ina226_power_test3}
\end{figure}

\begin{figure}[hbt]
  \centering
  \includegraphics[width=0.9\linewidth]{figures/test3/ina226_current.png}
  \caption{Measured current (mA) during Test \#3 execution.}
  \label{fig:ina226_current_test3}
\end{figure}

\begin{figure}[hbt]
  \centering
  \includegraphics[width=0.9\linewidth]{figures/test3/ina226_voltage.png}
  \caption{Measured voltage (mV) during Test \#3 execution.}
  \label{fig:ina226_voltage_test3}
\end{figure}

\begin{figure}[hbt]
  \centering
  \includegraphics[width=\linewidth]{figures/test3/xpe.png}
  \caption{Theoretical power breakdown from XPE for Test \#3.}
  \label{fig:xpe_test3}
\end{figure}

\begin{figure}[hbt]
  \centering
  \includegraphics[width=\linewidth]{figures/test3/xpe_graphs.png}
  \caption{Theoretical power and static current from XPE for Test \#3.}
  \label{fig:xpe_graph_test3}
\end{figure}

\FloatBarrier
\suppressfloats[t]

\subsubsection*{Kernel and Dataflow Architecture}

% TOP HALF OF IMAGE
\begin{figure}[!ht]
  \centering
  \begin{adjustbox}{max totalsize={\textwidth}{0.72\textheight},
                    clip,trim=0 {.5\height} 0 0}
    \includegraphics{figures/test3/pl_stride_32_img_recon_kern.png}
  \end{adjustbox}
  \captionsetup{aboveskip=6pt,belowskip=6pt,font=footnotesize}
  \caption{AI Engine graph of 32 Image Reconstruction kernels within a single \textit{bp\_cluster}. (top)}
  \label{fig:aie_graph_test3}
\end{figure}

\clearpage

% BOTTOM HALF OF IMAGE
\begin{figure}[p]\ContinuedFloat
  \centering
  \begin{adjustbox}{max totalsize={\textwidth}{0.85\textheight},
                    clip,trim=0 0 0 {.5\height}}
    \includegraphics{figures/test3/pl_stride_32_img_recon_kern.png}
  \end{adjustbox}
  \caption{(continued)}
\end{figure}

\begin{figure}[hbt]
  \centering
  \includegraphics[width=\linewidth]{figures/test3/kernel_tile_layout.png}
  \caption{Kernel placement in the AI Engine array.
  Each \texttt{bp\_cluster} is confined to a 8x7 grid.}
  \label{fig:kernel_tile_layout_test3}
\end{figure}

\begin{figure}[hbt]
  \centering
  \includegraphics[width=\linewidth]{figures/test3/kernel_tile_dataflow.png}
  \caption{Kernel-to-kernel dataflow connections in AI Engine fabric.}
  \label{fig:kernel_tile_dataflow_test3}
\end{figure}

