% =============================================================================
\section{Introduction}
% =============================================================================
\label{sec:intro}

% -----------------------------------------------------------------------------
% Versal Architecture
% -----------------------------------------------------------------------------
\begin{figure}[hbtp]
  \begin{center}
    \includegraphics[width=\textwidth]
    {figures/versal_arch.png}
  \end{center}
  \caption{Versal ACAP Functional Diagram ~\cite{AMD_Versal_WP505}}
  \label{fig:versal_arch}
\end{figure}
% -----------------------------------------------------------------------------

Synthetic Aperture Radar (SAR) is a high-resolution imaging technique
that exploits the motion of a radar antenna over a target region to
synthesize a large effective aperture~\cite{Jansing_SAR_Concepts}. By 
transmitting pulses and recording the backscattered echoes at multiple
positions along its path, SAR systems collect data that can be
processed into detailed scene images. The key advantage of SAR is its 
ability to achieve fine resolution in both range and cross-range,
making it essential for remote sensing tasks such as Earth observation, 
surveillance, and environmental monitoring.

Among various SAR image formation methods, backprojection is a
conceptually straightforward yet computationally intensive approach.
It coherently sums the radar returns for each image pixel, applying
appropriate time shifts (or phase corrections) to align signal energy
across the collected pulses. This effectively recreates the reflected
energy distribution, yielding high-quality images resilient to motion
errors, off-nominal geometries, and wide-band signals. However,
backprojection has a computational load proportional to the product of
the number of pulses and the number of pixels, which can be expensive
to implement at scale.

This work presents the design and implementation of a SAR backprojection
pipeline on the Versal ACAP (Adaptive Compute Acceleration Platform) from
AMD. The Versal ACAP is a heterogeneous compute environment featuring
Scalar Engines (CPUs), Adaptable Engines (FPGA/programmable logic), and
Intelligent Engines (AI and DSP cores), which allow developers to
accelerate compute-heavy tasks \cite{AMD_Versal_WP505}. 

This current implementation is designed to evaluate the performance
potential of the AI Engine (AIE) in SAR backprojection. It primarily
relies on AIE processing, with the ARM Cortex-A72 processor managing
execution flow and kernel configurations for the AIE. This approach
isolates and analyzes the AIE's contribution without involving other 
processing resources like the FPGA or DSP engines. Future work may explore
utilizing these additional resources. Currently, the implementation is
in its infancy and focuses on capturing performance metrics for a
single pulse at various azimuth image lines. While multi-pulse
processing is possible within the current framework, full validation of
the results across multiple pulses is challenging due to the large
volume of data generated.

