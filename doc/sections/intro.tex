% =============================================================================
\section{Introduction}
% =============================================================================
\label{sec:intro}

% -----------------------------------------------------------------------------
% Versal Architecture
% -----------------------------------------------------------------------------
\begin{figure}[hbtp]
  \begin{center}
    \includegraphics[width=0.85\textwidth]
    {figures/sar-flight-path-geometry-data.png}
  \end{center}
  \caption{Scanning configuration for a left looking SAR with a rectangular antenna~\cite{Olmsted1993}}
  \label{fig:sar-flight-path-geometry-data}
\end{figure}
% -----------------------------------------------------------------------------

Synthetic Aperture Radar (SAR) is a radar imaging modality that enables
high-resolution imagery by using antenna motion over the scene and combining
measurements from many nearby positions along the flight path, shown in figure
\ref{fig:sar-flight-path-geometry-data}. At centimeter wavelengths, achieving
about ten meter resolution on the ground with a single antenna would require an
antenna several kilometers long from orbit and still tens to a few hundred
meters from typical aircraft, which is not realistic to build or fly. During
processing, the many measurements are combined so they behave as if they were
taken with a much longer antenna, which provides the needed resolution without
a very large physical antenna. Because SAR does not rely on sunlight, it can
image day or night and in most weather conditions, supporting Earth
observation, hazard assessment, sea-ice tracking, wetland mapping, and broader
environmental monitoring~\cite{NASAEarthdataSAR}.

A few geometry terms appear throughout this paper. The flight direction is the
along-track or azimuth direction and the perpendicular direction is the
across-track or range direction. The distance measured along the radar line of
sight is the slant range, which can be projected to ground range. The look
angle describes how far the antenna points away from nadir, while the incidence
angle is the angle between the radar beam and the local surface normal.
Depending on terrain and viewing geometry, SAR images can contain layover and
shadow.

The work throughout this paper focuses on spotlight-mode backprojection, an
image reconstruction technique that operates on the raw complex radar
measurements known as phase-history data. Stripmap-mode backprojection is left
for future work (see Section~\ref{sec:stripmap_mode} for more details). For
each image pixel, backprojection uses the recorded platform trajectory to
compute the two-way travel time to that pixel for every pulse. Each recorded
echo is aligned to that pixel by applying the computed delay. This can be done
by shifting the waveform in time or by applying the corresponding phase shift
across the signal’s bandwidth. After alignment, the echoes are summed. When the
delays match the true geometry and timing, the aligned echoes add in phase at
that pixel and tend to cancel elsewhere, producing a focused image. Because
backprojection coherently combines many pulses for every pixel, the echoes from
the true location align in phase and reinforce, increasing the pixel’s
intensity even when the platform does not fly a perfectly straight path or the
viewing geometry is not ideal, provided the trajectory and timing are modeled
with sufficient accuracy. These strengths come with significant computational
cost. Consequently, many operational systems transmit phase-history data to the
ground and form images there rather than on board.

% -----------------------------------------------------------------------------
% Versal Architecture
% -----------------------------------------------------------------------------
\begin{figure}[hbtp]
  \begin{center}
    \includegraphics[width=0.85\textwidth]
    {figures/versal_arch.png}
  \end{center}
  \caption{Versal ACAP Functional Diagram ~\cite{AMD_Versal_WP505}}
  \label{fig:versal_arch}
\end{figure}
% -----------------------------------------------------------------------------

This paper develops an On-Board Processing (OBP) implementation of
backprojection on a Versal Adaptive Compute Acceleration Platform (ACAP), as
shown in figure \ref{fig:versal_arch}. The Versal ACAP is a heterogeneous
compute environment featuring Scalar Engines (CPUs), Adaptable Engines
(FPGA/programmable logic), and Intelligent Engines (AI and DSP cores), which
allow developers to accelerate compute-heavy tasks \cite{AMD_Versal_WP505}. We
design and integrate a data path from the dual Cortex-A72 processors, through
the network-on-chip, to the AI Engines, with ping-pong buffering, DMA
stride-controlled reordering to feed the AI Engine and Programmable Logic (PL)
kernels efficiently. The FPGA fabric hosts the DMA reordering logic, and the AI
Engines perform the core backprojection computations. We also provide
simulation code for the AI Engine array and testbenches FPGA logic. The
Cortex-A72 processors orchestrate the launch of the spotlight-mode
backprojection benchmark and record execution times. Upon completion, the AI
Engines stream the image through the PL DMA reordering kernel to assemble a
contiguous image in DDR. We report execution times and representative power
measurements at selected AI Engine core counts and illustrate how the design
scales with additional cores. These results characterize the performance and
power trade space for on-board backprojection on this class of device.


