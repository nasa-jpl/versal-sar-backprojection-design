% =============================================================================
\section{Implementation}
% =============================================================================
\label{sec:impl}

% =============================================================================
\subsection{Overview}
% =============================================================================

% -----------------------------------------------------------------------------
% AIE Graph
% -----------------------------------------------------------------------------
\begin{figure}[hbtp]
  \begin{center}
    \includegraphics[width=\textwidth]
    {figures/aie_graph.png}
  \end{center}
  \caption{AIE Graph Visualization (created with Vitis Analyzer).}
  \label{fig:aie_graph}
\end{figure}
% -----------------------------------------------------------------------------

The SAR Backprojection implementation employs the spotlight imaging mode
and leverages all three levels of parallelism offered by the Versal ACAP: 
SIMD, multicore, and ILP. Due to time constraints, the current design 
focuses primarily on SIMD and multicore parallelism; future work will
explore incorporating more ILP for further performance optimization.

Several custom AIE kernel designs with various graph constructs were iterated
upon. The design merged to implementing the following two custom AIE kernels:
Slowtime Splicer and Image Reconstruction kernels. The design supports scaling
the number of Image Reconstruction kernels between 4 and 8, resulting in a
total AIE kernel count of 5 or 9 respectively. These kernels play a crucial
role in accelerating the computationally demanding steps of the Backprojection
algorithm. Additionally, the graph model heavily relies on broadcasting, which
is a AIE graph construct that allows broadcasting the same data to multiple AIE
tiles at the same time.

\begin{itemize}
    \item \textbf{Slowtime Splicer Kernel:} A dedicated AIE kernel responsible 
        for splicing together slowtime data, comprising antenna x, y, and z 
        coordinates, as well as range to scene center. This data is received 
        from the ARM processor and then broadcasted to multiple image
        reconstruction kernels.
    \item \textbf{Image Reconstruction Kernels:} These kernels are also 
        implemented on AIEs and handle the core backprojection processing. They
        calculate the differential range for each pixel in the image (m),
        measuring its relative distance compared to the central pixel. They 
        perform cumulative summing and phase correction, leveraging the 
        parallel processing capabilities of the AIEs for efficient execution.
\end{itemize}

% =============================================================================
\subsection{Data Set}
% =============================================================================

The data set used is the "Gotcha Volumetric SAR Data Set, Version 
1.0"~\cite{GOTCHA}. This data set consists of SAR phase history
data collected at X-band with a 640 MHz bandwidth with full azimuth coverage
at 8 different elevation angles with full polarization. The imaging scene
consists of numerous civilian vehicles and calibration targets. The data is
stored in MATLAB binary format (*.mat files). Each file contains the phase
history collected over one degree of azimuth for a single pass and a single
polarization. Loading a file gives a single MATLAB structure with fields
containing the k-space data, frequencies, x,y,z coordinate antenna locations,
range to scene center, azimuth angle (degrees), and elevation angle (degrees).


% =============================================================================
\subsection{Computational Elements}
% =============================================================================

The design relies on AIE processing, with the ARM Cortex-A72 processor managing
AIE input, execution flow and kernel configurations for the AIE. This approach
isolates and analyzes the AIE's contribution without involving other processing
resources like the FPGA or DSP engines. Future work may involve utilizing these 
additional resources. 




