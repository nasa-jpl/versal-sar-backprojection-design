% =============================================================================
\section{MATLAB Code}
% =============================================================================
\label{sec:matlab_appendix}

The MATLAB implementation (shown in listing~\ref{lst:matlab_code}) is used
primarily to generate and manipulate the phase history data serving as input to
the Versal platform. Using Gorham’s SAR dataset and mathematical
formulation~\cite{Gorham}, 

The MATLAB implementation (shown in Listing~\ref{lst:matlab_code}) is primarily
used to generate and manipulate the phase history data that serve as input to
the Versal platform. Using Gorham’s SAR dataset and mathematical
formulation~\cite{Gorham}, the MATLAB code produces post-processed iFFT'd
time-domain data after zero-padding in the frequency domain to achieve the
desired range resolution. The parameter \texttt{data.Nfft} controls the number
of range-frequency samples used in this process: by default it is set to 424,
matching the actual number of measured range samples in the dataset. Increasing
\texttt{Nfft} beyond 424 effectively zero-pads the spectrum, resulting in finer
range resolution when the data are transformed to the time domain via the iFFT.
This preprocessing step ensures that the data provided to the Versal hardware
accurately represent the intended input conditions while removing the need for
on-chip FFT and iFFT operations. The resulting MATLAB images also serve as
reference outputs for validating the accuracy of the AI Engine backprojection
results (shown in Figures~\ref{fig:matlab_img_360} and~\ref{fig:matlab_img_5}).
For example, setting \texttt{FileNum = 5} limits processing to five input
files, corresponding to a 5° aperture around the scene center.

\begin{figure}[!htb]
  \centering
  \begin{subfigure}{0.48\linewidth}
    \centering
    \includegraphics[width=\linewidth, trim={0 20bp 0 10bp},
    clip]{figures/MATLAB_GOTCHA_360Deg_424Nfft.jpg}
    \caption{For 360° coverage with 424 range-compressed samples
    (\texttt{FileNum = 360}, \texttt{data.Nfft = 424}).}
    \label{fig:matlab_img_360}
  \end{subfigure}\hfill
  \begin{subfigure}{0.48\linewidth}
    \centering
    \includegraphics[width=\linewidth, trim={0 20bp 0 10bp},
    clip]{figures/MATLAB_GOTCHA_5Deg_512Nfft.jpg}
    \caption{For 5° coverage with 512 range-compressed samples (\texttt{FileNum
    = 5}, \texttt{data.Nfft = 512}).}
    \label{fig:matlab_img_5}
  \end{subfigure}
  \caption{Generated MATLAB images from the GOTCHA dataset at two aperture
  extents.}
  \label{fig:matlab_img_side_by_side}
\end{figure}


\begin{lstlisting}[
  style=Matlab-editor,
  caption={MATLAB script for preprocessing input data and generating reference
  output imagery for Versal comparison. The function definition of
  \texttt{bpBasic(...)} is provided from the MATLAB code within "SAR Image
  Formation Toolbox for MATLAB" paper by Gorham et al. The
  \texttt{dlmwrite(...)} functions provide the preprocessed CSV input for
  Versal. \cite{Gorham}},
  label={lst:matlab_code},
  basicstyle=\ttfamily\footnotesize,
  captionpos=b
]
% Author: Austin Owens
% Email: austin.t.owens@jpl.nasa.gov
%
% data.Nfft: Size of the FFT to form the range profile
% data.deltaF: Step size of frequency data (Hz)
% data.minF: Vector containing the start frequency of each pulse (Hz)
% data.x_mat: The x-position of each pixel (m)
% data.y_mat: The y-position of each pixel (m)
% data.z_mat: The z-position of each pixel (m)
% data.AntX: The x-position of the sensor at each pulse (m)
% data.AntY: The y-position of the sensor at each pulse (m)
% data.AntZ: The z-position of the sensor at each pulse (m)
% data.R0: The range to scene center (m)
% data.phdata: Phase history data (frequency domain)
%              Fast time in rows, slow time in columns
%
% data definitions:
% i.) data.fp: Phase history data (complex)
% ii.) data.freq: Column vector containing frequencies (Hz)
% iii.) data.x: Row vector containing the x-coordinate of the antenna location
% iv.) data.y: Row vector containing the y-coordinate of the antenna location
% v.) data.z: Row vector containing the z-coordinate of the antenna location
% vi.) data.r0: Row vector containing range to scene center from the antenna location
% vii.) data.th: Row vector containing azimuth angle. 0 degrees is the pos x-axis
% viii.) data.phi: Row vector containing elevation angle. 0 degrees is the xy-plane
% ix.) data.af: Structure containing autofocus information (only for HH and VV data). 
%      It contains 2 row vectors:
% x.) data.af.r_correct: Row vector containing the correction for r0
% xi.) data.af.ph_correct: Row vector containing the phase correction

clc;clear;

data.AntX = [];
data.AntY = [];
data.AntZ = [];
data.R0 = [];
data.phdata = [];
avg_delta_freq = 0;
min_freq = 0;
polarization = "HH";

% Number of files to process. Each file represents 1 degree (360 deg total)
FileNum = 360;

for n = 1:FileNum   
    str = sprintf('/mnt/disk/austin/SAR_DATA/GOTCHA-CP_Disc1/DATA/pass1/%s/data_3dsar_pass1_az%03d_%s.mat', polarization, n, polarization);
    az = load(str);
    data.AntX = [data.AntX, az.data.x];
    data.AntY = [data.AntY, az.data.y];
    data.AntZ = [data.AntZ, az.data.z];
    data.R0 = [data.R0, az.data.r0];
    data.phdata = [data.phdata, az.data.fp];
    avg_delta_freq = avg_delta_freq + mean(diff(az.data.freq));
    min_freq = az.data.freq(1);
end
data.deltaF = avg_delta_freq/FileNum;

data.Nfft = 424;
data.minF = ones(size(data.AntX))*min_freq;
c = 299792458;
maxWr = c/(2*data.deltaF);
res = maxWr/data.Nfft;
data.res = res;
x_mat = -maxWr/2:res:(maxWr/2)-res;

AntAz = unwrap(atan2(data.AntY,data.AntX));
deltaAz = abs(mean(diff(AntAz)));
totalAz = max(AntAz) - min(AntAz);
dx = c/(2*totalAz*mean(data.minF));
maxWx = c/(2*deltaAz*mean(data.minF));
y_mat = -maxWx/2:dx:(maxWx/2);

[data.x_mat, data.y_mat] = meshgrid(x_mat, y_mat);
data.z_mat = zeros(size(data.x_mat));

bp = bpBasic(data);
imagesc(x_mat, y_mat, 10*log10(abs(bp.im_final))); % Show image
axis('xy'); % Flip axis
colormap('jet'); % Use diffrent color mapping
caxis(caxis + 30); % Shift color scale
colorbar; % Show color scale

% Used to generate Versal input for time domain range compressed samples
% NOTE: Assumes FileNum = 360 for full 360 degree coverage
%dlmwrite("gotcha_phdata_64-out-of-424-rc-samples_pass1_360deg_HH.csv", fftshift(ifft(data.phdata, 64), 1)', 'delimiter', ',', 'precision', 12)
%dlmwrite("gotcha_phdata_128-out-of-424-rc-samples_pass1_360deg_HH.csv", fftshift(ifft(data.phdata, 128), 1)', 'delimiter', ',', 'precision', 12)
%dlmwrite("gotcha_phdata_256-out-of-424-rc-samples_pass1_360deg_HH.csv", fftshift(ifft(data.phdata, 256), 1)', 'delimiter', ',', 'precision', 12)
%dlmwrite("gotcha_phdata_512-out-of-424-rc-samples_pass1_360deg_HH.csv", fftshift(ifft(data.phdata, 512), 1)', 'delimiter', ',', 'precision', 12)
\end{lstlisting}

\clearpage

% =============================================================================
\section{Revision History}
% =============================================================================
\label{sec:rev_history_appendix}

\begin{table}[h]
%\caption{Revision History}
\begin{center}
\begin{tabular}{|c|p{100mm}|}
\hline
Date & Description\\
\hline\hline
&\\
10/07/2025  & Created the document.\\
&\\
\hline
\end{tabular}
\end{center}
\end{table}


